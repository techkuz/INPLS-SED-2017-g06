\chapter{Introduction}
\label{chap:introduction}


\section{Overview}
This software system is aimed at informing of a car crash crisis situation in
order to allow for crisis handling. At anytime and anywhere, anyone can be the
witness or victim of a car crash and might be in a situation allowing for
alerting this crisis. This system has for objectives to support crisis
declaration and secure administration and crisis handling by the system
professional users.\\
This version of the document contains all previous changes as as well as new
functions:
\begin{itemize}
  \item Authentication (For authenticated users, after three failing
  connection tries a captcha test is proposed. A password reset is always
  possible)
  \item Information Diffusion(Some carefully selected information on
the crisis and alerts is available to the familly members for its diffusion)
  \item Time (Some timing statistics for all events/operations
are made available to the administrator in a abstract and detailed manner)
\end{itemize}
  




\section{Purpose and recipients of the document}
This document is a design document. The aim of this document is to provide an
example of how the design of a particular software system should be documented. 

The recipient of this document is the development company (ADC) in charge
of delivering the software system. The company's developers are
expected to use this document as the basis for carrying out the actual
development and deployment of the product (i.e. implementation, testing
and maintenance).





\section{Definitions, acronyms and abbreviations}
Framework - is an abstraction in which software providing generic functionality
can be selectively changed by additional user-written code, thus providing application-specific software.


  
\section{Document structure} 
This document is organised as follows: Section \ref{chap:AM} provides a general
overview of the main concepts gathered during the analysis phase, in particular those concerning the software system abstract
types, as well as the actors that interact with the
software system through their interfaces. 

The technologies used not only during the design and development phase, but 
also those required to make the software system runnable are presented in
Section~\ref{chap:techFrm}.

The architecture of the software system to be implemented and deployed is
described in Section \ref{chap:arch}. This section presents the components of
the software system architecture along with their interactions, both from the static and dynamic viewpoints.

The detailed design of each \gls{system operation} is given in Section
\ref{chap:detDesign}, whereas Section \ref{chap:know_limitations} enumerates the
current limitations of the software system at the writing time of this document.

Next, Section \ref{chap:testing} presents the different test cases used to
verify the correct behaviour of the software system's functional and not functional requirements.

Finally, Section \ref{chap:final_conclusion} draws the
conclusion achieved during the design and implementation of the software system.
 
 