\chapter{Introduction}
\label{chap:introduction}


\section{Scope}
The intention is that this document will provide a comprehensive guide, enabling
users to use iCrash to its full extent. It is in three parts.\\

Usage Guide which is aimed at describing the general use of the software since
it is deployed, configured and run. Special attention is paid to actors which
rely on the software to perform a set of business activities (procedures) aimed at reaching a particular goal. Procedures are split into two groups: multi-procedures and mono-procedures.\\

Software Operations which explain allowed software operations using parameters,
pre- and post-conditions, output messages, triggering.\\

Error messages and problem resolutions which lists and explains in detail all
problems that might have arisen while using the software. \\

This document does not provide details about relationships with software
stakeholders as it is described in additional software licence agreement. It
also contains limited information of copyright and trademark notices.\\

This document is not intended to be final as updates and corrections are planned
to be added later. \\

This document may be used with iCrash Design Document, deployment iCrash
version, development iCrash version.\\


%This document \gls{KuzmaTerm}  provides \ldots
%Example: This document provides minimum acceptable information for knowing how
% to use the software system \mysystemname.


This document does not \ldots 
 
This document is not \ldots
%Example: This document is not intended to provide information about how to
% connect, deploy, configure, or use any external device or
% third-party software system that is rqeuired for the correct funcitoning of
% \mysystemname.

 
This document may be used with \ldots
%This document may be used with other documents provided by third-party
% companies to have an overall view and correct understanding of the environment
% and procedures where the software system \mysystemname is aimed to be deployed
% and run.




\section{Purpose}
In this section you explain the purpose (i.e. aim, objectives) of the user's
manual. In the following some examples of opening statements to be used in this
section. W  

The purpose of this document is \ldots

This document defines \ldots

This document is meant to \ldots



\section{Intended audience}
Description of the categories of persons targeted by this document together with the description of how they are expected to exploit the content of the document.


\section{\MySystem(v1.0)}
The iCrash system belongs to the Crisis Management Systems Domain. It is a
system dedicated to crisis professional and non professional end users. It has
to be considered as an autonomous and external service for the society. It is not an institutional system certified and guaranteed by any governmental entity and thus, must be used with caution.\\


\subsection{Actors \& Functionalities}
Overview of all the \textbf{\emph{\glspl{actor}}} interacting with the software
being them either humans (called end-users in the standard
\cite{IEEE-2001-userdocumentation}) or not. For each actor, describe the main
software functions that are offered to him. Structure of this sub-section MUST
be by actor/functionalities.
\begin{itemize}
\item \emph{actComCompany}: Communication Company 
	\begin{itemize}
	  \item Delivering any SMS sent by any human to the iCrash’s phone number.
	  \item Transmit SMS messages from the ABC company that owns the iCrash system
	  to any human having an SMS compatible device accessible using aphone number.
	\end{itemize}
\item \emph{actAdministrator}: Administrator
	\begin{itemize}
	  \item  Adding or deleting coordinator actors from the system and its
	  environment.
	\end{itemize}
\item \emph{actCoordinator}: Coordinator
	\begin{itemize}
	  \item Monitoring the existing alerts and crisis.
	  \item Managing alerts and crisis until their termination.
	\end{itemize}
\item \emph{actActivator}: Activator
	\begin{itemize}
	  \item Communicate the current time to the system.
	  \item Notify the administrator that some crisis are still pending for a too
	  long time.
	\end{itemize}
\item \emph{actMsrCreator}: Creator
	\begin{itemize}
	  \item Installing the iCrash system.
	  \item Defining the values for the initial system’s state.
	  \item Defining the values for the initial system’s environment.
	  \item Ensuring the integration of the iCrash system with its initial
	  environment.
	\end{itemize}
\end{itemize}



\subsection{Operating environment}
The iCrash application is supposed to be deployed over at least 3 different
computers: 1 database server, 1 application server, and at least 1 client.\\

However in order to ease deployment process deployment via Vagrant is possible.
It requires single PC to deploy and run iCrash.\\

Vagrant (author rights described in Product Information - Copyright section) is
an open-source software product for building and maintaining portable virtual
development environments. It is used to demonstrate the application working using users machine, servers application and database machine.\\

Another application required for successful deployment of iCrash is VirtualBox
(ver. 5.1.x). \\

\section{Document structure}  
Information on how this document is organised and it is expected to be
used. Recommendations on which members of the audience
should consult which sections of the document, and explanations about the used
notation (i.e. description of formats and conventions) must also be provided.





